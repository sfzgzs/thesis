\begin{frame}{Result 2}
    \pause
    \vspace{-1.5em}
    \begin{ntheorem}[Acceptability Theorem: Part 2](2.2)
        Any elementary function is effectively locally uniformly continuous \textbf{\textcolor{BrickRed}{w.r.t.\null{} an effective open exhaustion}} for its domain.
    \end{ntheorem}
    \pause
    \vspace{-0.5em}
    \begin{exampleblock}{\color{OliveGreen}\textbf{(We prove that this is) equivalent to proving:}}
        Any elementary function has a local continuity witness.
    \end{exampleblock}
    \pause
    \vspace{-0.5em}
    \begin{definition}[Local continuity witness]
        Let $f:\Reals\to\Reals$. A recursive function $N:\Rationals\times\Rationals\times\Nats\to\Nats$ is called a \textbf{local continuity witness} for $f$\index{local continuity witness} iff for any $a,b\in\Rationals$ with $[a,b]\subseteq\dom(f)$ and $k\in\Nats$, we have
        \[ 
            \forall x,y \in (a,b) \quad \abs{x-y}<2^{-N(a,b,k)} \then \abs{f(x)-f(y)}<2^{-k}.
        \] 
    \end{definition}
    \vspace{-0.5em}
    \pause
    \begin{theorem}
        Let $f:\Reals\to\Reals$ be \WhileCC-approximable and monotone on its domain. 
        Then $f$ has a local continuity witness.
    \end{theorem}
    \pause
        {\textbf{\textcolor{OliveGreen}{The notion of effective local uniform continuity is independent
                of effective open exhaustion.}}}
    \note{Now we need to prove effective local uniform continuity of any elementary function with respect to its domain's effective open exhaustion.
    It is easier for our purposes to use an equivalent definition.
    Intuitively, the existence of a local continuity witness is similar to effective local uniform continuity w.r.t. an effective open exhaustion, except that, instead of the recursive function parameterized by the stage number, it is parameterized by the endpoints of an interval in its domain.\\
    The proof for both base cases and induction steps use the \WhileCC-approximability theorem (proved earlier) to overapproximate how much the value of a function changes within an interval in the domain.
    The composition case is mindnumbingly technical, so we won't go into the details here.
    %The composition case is a little bit trickier, since we need to prove that for any elementary function $f$ and open set $U$ with an effective exhaustion, we have some effective way of computing the stage number of $f(x)$ for any $x$ in an interval in some stage $l$ of an effective open exhaustion for $f^{-1}(U)$. But I won't go into the details here.
    }

\end{frame}