
\begin{frame}{Background -- Acceptability}
\pause
    \begin{definition}[\textbf{\textcolor{violet}{Acceptability}}]
        A function $f:\Reals\to\Reals$ is \textbf{\textcolor{violet}{acceptable}} if there exists a sequence $X$ where:
        \begin{enumerate}
            \pause\item $X$ is an  \textbf{\textcolor{olive}{effective open exhaustion}} for $\dom(f)$, and
            \pause\item $f$ is  \textbf{\textcolor{brown}{effectively locally uniformly continuous w.r.t.\null{} $X$}}.
        \end{enumerate}
    \end{definition}
    % \vspace{-1.2em}
    \note[item]{First, let's start with defining acceptability formally.
    A function is called acceptable if its domain has an effective open exhaustion,
    and is effectively locally uniformly continuous with respect to that open
    exhaustion. Now, what are effective open exhaustions?
    and what does it mean for a function to be effectively locally uniformly
    continuous with respect to an effective open exhaustion?
    We will now define the concepts introduced here.}
\end{frame}
\begin{frame}{Background -- Effective Open Exhaustions}
\vspace{-1em}
\pause
\begin{definition}[\citep{ModelOfCompForPartFunc_MingQuanFuAndJeffZucker}]
        \pause A sequence $(U_1, U_2, \ldots)$ of open sets is called an \textbf{\textcolor{olive}{effective open exhaustion}} for an open $U\subseteq \Reals$ if 
        \pause \begin{enumerate}
                \item $ U = \bigcup_{l=0}^\infty U_l$, and
                \pause\item for each $l \in \Nats$,
                  $U_l$ \pause is a finite union of non-empty open finite intervals $I_1^l, I_2^l,...,I_{k_l}^l $ whose closures are pairwise disjoint, and
                \pause\item for each $l \in \Nats$,
                   $\overline{U_l} = \bigcup_{i=1}^{k_l} \overline{I_i^l} \subseteq U_{l+1}$.
                \pause\item 
                for all $l$, the components $I_i^l$ that are intervals building up the stage $U_l$, are \textit{rational} \pause and \textit{ordered} i.e., $I_i^l = (a_i^l, b_i^l)$ for some $a_i^l, b_i^l \in \Rationals$ where $b_i^l < a_{i+1}^l$ for $i=1,...,k_l-1$, and
                \pause\item the map
                $l\mapsto ( a_1^l , b_1^l, ...,
                 a_{k_l}^l, b_{k_l}^l)$
                which delivers the sequence of stages $U_l = I_1^l\cup ...\cup I_{k_l}^l$ is recursive.
        \end{enumerate}
    \end{definition}
    \pause
    \begin{exampleblock}{\textbf{Example}}
        The sequence of open sets $(-1, 1), (-2, 2), \ldots, (-k, k), \ldots$
        is the standard effective open exhaustion for $\Reals$.
    \end{exampleblock}
    
   
    \note[item]{What is an effective open exhaustion? 
    First of all, it's defined for open sets, so only open sets can have
    open exhaustions.}
    \note[item]{An open exhaustion is a sequence of open sets that all unioned
    together should give us the original open set.}
    \note[item]{each element of the sequence, $U_l$,is called a stage,
    and is a union of a bunch of open intervals, the closures of which are disjoint.}
    \note[item]{the closure of each stage falls under the next stage}
    \note[item]{In each stage, the intervals are ordered and the endpoints are rational.}
    \note[item]{And the reason we call these effective is becuase the map delivering the endpoints of intervals in a stage should be recursive.}
    \note[item]{Now, as an example for an effective open exhaustion, consider the sequence here. In this example, each stage consists of a single interval and the union of all these sets covers the entire $\Reals$.}
\end{frame}
\begin{frame}{Background -- Effective Local Uniform Continuity}
    \begin{definition}[\citep{ModelOfCompForPartFunc_MingQuanFuAndJeffZucker}]
        A function $f$ on $U$ is \textbf{ effectively locally uniformly continuous \textbf{\textcolor{BrickRed}{w.r.t.\null{} an effective open exhaustion}}} $(U_n)_{n \in \Nats}$ of $U$, \pause if there is a recursive function $M:\Nats^2\totalTo \Nats$ such that for all $k,l \in \Nats$ and all $x,y \in U_l$:
        \[
        \abs{x-y}<2^{-M(k,l)} \then \abs{f(x)-f(y)}<2^{-k}
        \]    
    \end{definition}
    \note[item]{Now, we also needed effective local uniform continuity with respect to an effective open exhaustion.
    
    This is basically defined as standard uniform continuity, but over each stage. Also we require that the modulus of continuity $M$ is computable. This technically lets us compute bit by bit how close $y$ should be to $x$, to enable to compute $f(y)$ instead of $f(x)$ to get $k$-many correct bits.}
\end{frame}